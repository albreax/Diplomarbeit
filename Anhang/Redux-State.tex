% !TeX root = ./Redux-State.tex
% config.tex

\author{Albrecht Naumann}
\title{Diplomarbeit}

\documentclass [
    a4paper,
    12pt,
    oneside
] {scrreprt}
% TODO: wenn notwendig, noch Fußnotengröße setzen -2pt

%
% Language
%
\usepackage[utf8]{inputenc}
\usepackage[T1]{fontenc} % europäische Zeichen

%
% Document Format
%
\usepackage{geometry}
\usepackage{graphicx}
\usepackage{subcaption}
\usepackage{float}

\usepackage[font=small,labelfont=bf]{caption}

% default pdf version 1.5 => changed to 1.7
\pdfminorversion=7

% Ränder
\geometry{includeheadfoot, top=2cm, bottom=2cm, left=3cm, right=2cm}
%Zeilenabstand
\renewcommand{\baselinestretch}{1.5}\normalsize
% Arial
% \renewcommand{\familydefault}{\sfdefault}
\tolerance=200
\emergencystretch=1em % Tolleranz

% Titelformat
% \usepackage{titlesec}
% \titleformat{\chapter}[block]
%   {\normalfont\LARGE\bfseries}{\thechapter.}{.5em}{\LARGE}
% \titlespacing*{\chapter}{0pt}{0pt}{1em}
% \titleformat*{\section}{\large\bfseries}
% \titlespacing*{\section}{0pt}{1em}{1em}
% \titleformat*{\subsection}{\normalfont\bfseries}

% global packages
%
\usepackage{url}
\usepackage[ampersand]{easylist}
\usepackage[utf8]{inputenc}
\usepackage{scrhack}
%
% references
%
\usepackage[babel,german=guillemets]{csquotes}
% \usepackage[style=authoryear]{biblatex}
\usepackage[
    backend=biber, 
    style=authoryear, 
    maxnames=2,
    minnames=1,
    isbn=false,
    doi=false,
]{biblatex} %biblatex mit biber laden
% \ExecuteBibliographyOptions{
%   sorting=nyt, %Sortierung Autor, Titel, Jahr
%   bibwarn=true, %Probleme mit den Daten, die Backend betreffen anzeigen isbn=false, %keine isbn anzeigen
%   url=false %keine url anzeigen
% }

\DefineBibliographyStrings{ngerman}{ 
   andothers = {et\addabbrvspace al\adddot},
   andmore   = {et\addabbrvspace al\adddot},
}

\addbibresource{literatur.bib}

\usepackage[table]{xcolor}
\definecolor{lstgrey}{rgb}{0.95,0.95,0.95}

% Code-Snippets 
\usepackage{listings}
\lstset{
    numbers=none, 
    numberstyle=\tiny,
    numbersep=5pt, 
    stepnumber=2, 
    frame=single,
    backgroundcolor=\color{lstgrey},
    basicstyle=\footnotesize\ttfamily,
    breaklines=true,
}

% DIR-Tree
\usepackage{dirtree}

% item list
%
\usepackage{paralist}
\renewcommand{\labelitemi}{$\bullet$}
\renewcommand{\labelitemii}{$\circ$}
\renewcommand{\labelitemiii}{$\cdot$}
\renewcommand{\labelitemiv}{$\ast$}

\usepackage{chngcntr} % => \counterwithin{figure}{chapter}
\counterwithin{figure}{chapter}

\usepackage[ngerman]{babel}
\addto\captionsngerman{
  % \renewcommand{\figurename}{Abbildung}
  % \renewcommand{\tablename}{Tab.}
  % \renewcommand{\lstlistingname}{Listing}
}
\renewcommand{\lstlistlistingname}{Abbildungsverzeichnis}%  List of Listings -> Abb.

\usepackage{amsmath}

% sonstiges
\usepackage{framed}
\usepackage{mathabx}

\usepackage{multicol}

% table
\usepackage{longtable}
\usepackage{tabularx,  multirow}
\usepackage{ltxtable}
 
% frame
\usepackage{efbox,graphicx}
\efboxsetup{linecolor=black,linewidth=.2pt}

\newcounter{magicrownumbers}
\newcommand\rownumber{\stepcounter{magicrownumbers}\arabic{magicrownumbers}}
\newcommand\rownumbereset{\setcounter{magicrownumbers}{0}}
% mute warnings
\usepackage{silence}
\WarningFilter{latex}{Writing or overwriting file}
\WarningFilter{latex}{File `Diplomarbeit_Albrecht_Naumann-storetable.tex' already exists on the system.}
%
% PDF output
%
\usepackage[
    bookmarks=true,         % show bookmarks bar?
    unicode=false,          % non-Latin characters in Acrobat’s bookmarks
    pdftoolbar=true,        % show Acrobat’s toolbar?
    pdfmenubar=true,        % show Acrobat’s menu?
    pdffitwindow=false,     % window fit to page when opened
    pdfstartview={FitH},    % fits the width of the page to the window
    pdfauthor={Albrecht Naumann},     % author
    pdfnewwindow=true,      % links in new PDF window
    colorlinks=false,       % false: boxed links; true: colored links
    hidelinks
    % linkcolor=red,          % color of internal links (change box color with linkbordercolor)
    % citecolor=green,        % color of links to bibliography
    % filecolor=magenta,      % color of file links
    % urlcolor=cyan           % color of external links
]{hyperref}


\title{Redux-State}
\begin{document}
\chapter*{Redux-State}
% 
\begin{filecontents}[overwrite]{\jobname-storetable.tex}
    
    \newcommand{\minitab}[2][l]{\begin{tabular}{#1}#2\end{tabular}}
    \begin{longtable}{l|X}
    \textbf{Schnittstelle} & \textbf{Beschreibung} \\ 
    \hline 
    
    \multirow{2}*{\minitab{IGuiState  \\ \lstinline|stores/guiState.ts|}}    & 
    Beschreibung des Zustands der grafischen Oberfläche des FreeDesign-Editors. 
    Beispiele hierfür sind die Darstellung von modalen Dialogen oder der Aufbau der Menüs. 
    
    \\
    
    \multirow{2}*{\minitab{IStageState     \\  \lstinline|stores/stageState.ts| }}      & 
    Vom Zustand der grafischen Oberfläche entkoppelt, wird durch die Schnittstellen \lstinline|IStageState| der Zustand der zentralen Oberfläche zur Produktbearbeitung (Stage) beschrieben.
    
    \\ 
    
    \multirow{2}*{\minitab{IProductState  \\   \lstinline|stores/productState.ts| }}   & 
    Die Zustandsbeschreibung des Produkts welches der Nutzer bearbeiten enthält, sowohl Information für Darstellung innerhalb des FreeDesign-Editors, als Informationen die für den Bestellprozess relevant sind.
    
    \\
    
    \multirow{2}*{\minitab{IDesignState   \\   \lstinline|stores/designState.ts| }}     & 
    Beschreibung des Designs welches der Nutzer bearbeiten. Des Weiteren wird durch diesen Zustand eine Historie der Bearbeitungsschritte gepflegt.
    
    \\
    
    \multirow{2}*{\minitab{IApiState  \\ \lstinline|stores/apiState| }}           & 
        Die bereits genannten Zustände, \lstinline|IProductState|, \lstinline|IDesignState|, \lstinline|IUserDesignState| und \lstinline|IUserState| werden von Werte belegt, die durch API bereitgestellt werden. Allerdings besitz der \emph{Redux-State} ein weiteres Unterobjekt, welches durch die Schnittstelle  \lstinline|IApiState| beschrieben wird und ebenfalls durch Daten der API beschrieben wird. Hierzu zählt eine Liste von Schriftarten, die innerhalb eines Designs genutzt werden können, sowie eine Liste der Bilder, die ein Kunde in seinem Design nutzt.  
    
    \\
    
    \multirow{2}*{\minitab{ICoreState   \\  \lstinline|@unitedprint/frontend-core|}} & 
        Der Zustand \lstinline|core| wird vom Quelletext nicht genutzt und wurde mutmaßlich vergessen bei einem Refactoring zu entfernen.
    
    \\
    
    \multirow{2}*{\minitab{IUserState   \\  \lstinline|@unitedprint/frontend-core| }} & 
        Die Struktur enthält Informationen des Nutzers, wie Kundenart oder Rechnungs- und Lieferaddresse. 
    
    \\
    
    \multirow{2}*{\minitab{ ITranslationsState \\ \lstinline|@unitedprint/frontend-core| }} & 
        \lstinline|ITranslationsState| enthält eine Liste der Texte für die grafische Oberfläche.
    
    \\
    
    \multirow{2}*{\minitab{IUserDesignState \\ \lstinline|stores/userDesignState|  }}   & 
    Für das Verwalten von Metadaten eines Kundendesigns, wie der Designname unter dem ein Design gespeichert wurde oder der Zeitstempel, wann ein Design erstellt, wird durch die Schnittstelle \lstinline|IUserDesignState| beschrieben.
    \\
    
    \end{longtable}
\end{filecontents}
\LTXtable{\linewidth}{\jobname-storetable.tex}

\end{document}