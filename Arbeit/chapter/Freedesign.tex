% freedesign.tex
\chapter{Der Freedesign-Editor}
% TODO: Anpassung
% In allen Online-Shops des Unternehmens wird der Freedesign-Editor, zur Gestalten einer Vielzahl der angeboten Produkten, bereitgestellt. Aktuell (März 2019) können ca. 1000 Einzelprodukte mit dem Editor gestaltet werden. Für die Gestaltung werden die Produkte maßstabsgerecht mit all ihren produktspezifischen Eigenschaften dargestellt. Dazu gehören unter anderem die Darstellung von Beschnitten, Stanzformen, Falzlinien oder Bereichen, die nicht bedruckbar sind, wie zum Beispiel Fenster bei Briefumschlägen.
% Es können vom Kunden Produkte mit Texten, geometrischen Formen oder Bildern gestaltet werden. Der Editor bietet die Möglichkeit, sowohl eigene Bilder per Bild-Upload einzubinden, als auch Bilder zu nutzen, die durch Unitedprint im Editor bereitgestellt werden. Für einige Produkte werden auch Designvorlagen angeboten, die als Startpunkt zur eigenen Gestaltung dienen können. Weiterhin stellt der Editor eine Speicherfunktion bereit, mit der gestaltete Produkte für eine spätere Bearbeitung gesichert werden können. Wurde vom Kunden eine Bestellung abgeschlossen, wird serverseitig automatisiert ein druckbares PDF des Entwurfs erstellt und dem folgenden Produktionsablauf beigefügt.
%
% Der Freedesign-Editor wird seit über 10 Jahren in verschiedenen Versionen in den Online-Shops des Unternehmens bereitgestellt. Zunächst wurde der Editor als Flash-Applikation angeboten, welche im Sommer 2017 durch ein Javascript-Applikation abgelöst wurde.
