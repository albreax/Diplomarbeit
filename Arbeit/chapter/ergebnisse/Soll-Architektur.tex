% Soll-Architektur.tex
\section{Soll-Architektur}
\subsection{Taxonomie der Programmnamen}
Die Namen der Programme die durch das FreeDesign-Projekt erstellt werden, folgen einem spezifischen Namensschema. Dieses setzt sich aus einem Programmnamen und einem Postfix, welches die Zielplattform angibt, zusammen. Der Programmnamen sollte möglichst aussagekräftig sein und besteht aus Kleinbuchstaben. Tabelle \ref{table:postfix} enthält Postfix-Möglichkeiten für die aktuell bekannten Zielplattformen.

\begin{table}[H]
    \centering
    \caption{Auflistung der Postfix-Möglichkeiten für die Programmnamen}
    \label{table:postfix}
    \begin{tabular}{r|l}
        \textbf{Prefix} & \textbf{Zielplattform} \\
        \hline
        \lstinline|web| & Internetbrowser \\
        \lstinline|cli| & Kommandozeile \\
        \lstinline|ios| & IOS \\
        \lstinline|android| & Android 
    \end{tabular}
\end{table}

Durch diese Konvention ergibt sich folgende Änderung der Namengebung: 
\begin{table}[H]
    \centering
    \caption{Änderungen der Bezeichnungen der Programme}
    \label{table:Programmnamen}
    \begin{tabular}{r|l}
        \textbf{alte Bezeichnung} & \textbf{neue Bezeichnung} \\
        \hline
        \lstinline|freedesign| & \lstinline|freedesign-web| \\
        \lstinline|draftImporterCli| & \lstinline|ai-design-converter-cli| \\
        \lstinline|designToSvgCLI| & \lstinline|fd-design-to-svg-converter-cli| 
    \end{tabular}
\end{table}

\subsection{nichtfunktionalen Anforderungen}
\begin{table}[H]
    \centering
    \caption{nichtfunktionale Anforderungen an die Architektur des FreeDesign-Projekt}. 
    \label{table:nfqa}
    \begin{tabularx}{\columnwidth}{c|X}
    \textbf{Merkmal} & \textbf{Teilmerkmale} \\
    \hline
    \hline
    \textbf{funktionale Eignung} & 
        funktionale Vollständigkeit {$\circ$} funktionale Korrektheit {$\circ$} funktionale Angemessenheit 
    \\
    \hline
    \textbf{Leistungseffizienz}  & 
        \begin{itemize}
            \item 
        \end{itemize}
        Zeitverhalten {$\circ$} Ressourcenauslastung {$\circ$} Kapazität
    \\
    \hline
    \textbf{Kompatibilität}  & 
        Koexistenz {$\circ$} Interoperabilität
    \\
    \hline
    \textbf{Benutzbarkeit}  & 
        Angemessenheit {$\circ$} Erkennbarkeit {$\circ$} Erlernbarkeit {$\circ$} Bedienbarkeit {$\circ$} Schutz vor Anwenderfehlern {$\circ$} Ästhetik der Benutzeroberfläche {$\circ$} Barrierefreiheit
    \\
    \hline
    \textbf{Zuverlässigkeit}  & 
        Laufzeit {$\circ$} Verfügbarkeit {$\circ$} Fehlertoleranz {$\circ$} Wiederherstellbarkeit
    \\
    \hline
    \textbf{Sicherheit}  & 
        Vertraulichkeit {$\circ$} Integrität {$\circ$} Unabstreitbarkeit {$\circ$} Rechenschaftspflicht {$\circ$} Authentizität 
    \\
    \hline
    \textbf{Wartbarkeit}  & 
        Modularität {$\circ$} Wiederverwendbarkeit {$\circ$} Analysierbarkeit {$\circ$} Modifizierbarkeit {$\circ$} Testbarkeit
    \\
    \hline
    \textbf{Portabilität} & 
        Anpassungsfähigkeit {$\circ$} Installationsmöglichkeiten {$\circ$} \newline Austauschbarkeit
\end{tabularx}
\end{table}

\subsection{technische Anforderungen}
% \subsection{Anforderungen an die Soll-Architektur} % Softwarearchitektur S. 108 - 

% \subsubsection{Funktionale Anforderungen}
% Die Tabelle \ref{table:fa} beschreibt die funktionalen Anforderungen die das FreeDesign-Projekt bereitstellen muss. Die Anforderungen sind unterteilt in \emph{FreeDesign-Webapp}, \emph{AI-Design-Converter-CLI} sowie \emph{SVG-Page-Renderer-CLI}.
% Der Teil \emph{FreeDesign-Webapp} bezieht auf die Webanwendung des FreeDesign-Editor und ermöglicht somit die Definition weiter grafischer Oberflächen des FreeDesign Editors. Denkbar hierfür wäre eine separate Oberflächen für native\footnote{Dies wäre durch die Nutzung von ReactNative möglich. \url{https://reactnative.dev}} Anwendung die sich an eine mobile Endgeräte richten. 
% \begin{filecontents}[overwrite]{\jobname-fa.tex}
%     \begin{longtable}{r@{\hspace{3mm}}lX}
%     \caption{Auflistung funktionaler Anforderungen}\\
%     \label{table:fa}
%     \rownumbereset
%     & \textbf{Anforderung} & \textbf{Beschreibung} \\
%     \hline
%     \hline
%     \multicolumn{3}{l}{\textbf{FreeDesign-Webapp}} \\
%     \hline
%     \rownumber & Darstellung Produktseite & Die Darstellung einer Produktseite ist abhängig vom Produkt und beinhaltet das Darstellen produktspezifischer Eigenschaften. \\
%     \rownumber & Darstellung Designs & Ein Design wird innerhalb einer ein Produktseite darzustellen. \\
%     \rownumber & Mehrsprachigkeit & Die Texte der grafische Oberfläche müssen für eine gegebene Landesprache übersetzt sein. \\
%     \end{longtable}
% \end{filecontents}
% \LTXtable{\linewidth}{\jobname-fa.tex}

% \subsubsection{Nichtfunktionale Anforderungen}
% \begin{filecontents}[overwrite]{\jobname-nfa.tex}
%     \begin{longtable}{r@{\hspace{3mm}}lX}
%     \caption{Auflistung der nichtfunktionalen Anforderungen}\\
%     \label{table:fa}
%     \rownumbereset
%     & \textbf{Anforderung} & \textbf{Beschreibung} \\
%     \hline
%     \hline
%     \multicolumn{3}{l}{\textbf{FreeDesign-Webapp}} \\
%     \hline
%     \rownumber & Darstellung Produktseite & Die Darstellung einer Produktseite ist abhängig vom Produkt und beinhaltet das Darstellen produktspezifischer Eigenschaften. \\
%     \rownumber & Darstellung Designs & Ein Design wird innerhalb einer ein Produktseite darzustellen. \\
%     \rownumber & Mehrsprachigkeit & Die Texte der grafische Oberfläche müssen für eine gegebene Landesprache übersetzt sein. \\
%     \end{longtable}
% \end{filecontents}
% \LTXtable{\linewidth}{\jobname-nfa.tex}

% Step 1. Anforderungen festlegen

