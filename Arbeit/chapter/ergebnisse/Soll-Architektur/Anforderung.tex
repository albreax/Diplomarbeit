\subsection{nichtfunktionalen Anforderungen}
Die nachfolgenden Qualitätsmerkmale sind einem Teilmenge aller Qualitätsmerkmale, die von der statischen Architektur des Projektes beeinflusst werden.   

\subsubsection{Leistungseffizienz}
Von der grafischen Oberfläche des FreeDesign-Editors wird erwartet, dass sie mit der geringst möglichen Latenz auf Eingaben durch den Nutzer reagiert. Die statische Architektur sollte diese nach Möglichkeit unterstützen.

\subsubsection{Kompatibilität}
Die Anbindung der Unitedprint-API sollte partiell und vollständig Austauschbar sein. 

\subsubsection{Benutzbarkeit}

\subsubsection{Zuverlässigkeit}

\subsubsection{Sicherheit}

\subsubsection{Wartbarkeit}
\paragraph{Änderungen der grafischen Oberfläche:}
Die grafischen Oberfläche des FreeDesign-Editors unterliegt stetiger Änderung und Erweiterung. 
Von diesen Änderungen sollten die Kernfunktionalitäten des Projektes bestmöglich geschützt sein. 

\paragraph{Unit-Testing:}
Die Strukturierung des Quelltextes soll das Testen mittels Unit-Test unterstützen. 

\paragraph{Unabhängigkeit von FreeDesign und Kommandozeilenprogramme:}
Um eine größtmögliche Unabhängigkeit des FreeDesign-Editors und der Kommandozeilenprogramme zu erreichen, sollte der gemeinsam genutzte Quelltext auf ein notwendiges Minimum reduziert werden.

\subsubsection{Portabilität}
Die Architektur soll die Bereitstellung des FreeDesign-Editors auf mehreren Zielplattformen ermöglichen. Neben der Bereitstellung der Anwendung als Single-Page-Application für die Internetpräsentation, wird IOS \& Android ebenfalls eine zukünftige Zielplattform sein. 
Auf eine eventuelle Forderung nach eine nativen Anwendung des FreeDesign-Editors für Windows, macOS und Linux sollte die Architektur ebenfalls vorbereitet sein. 

\subsection{technische Anforderungen}
% \subsection{Anforderungen an die Soll-Architektur} % Softwarearchitektur S. 108 - 

% \subsubsection{Funktionale Anforderungen}
% Die Tabelle \ref{table:fa} beschreibt die funktionalen Anforderungen die das FreeDesign-Projekt bereitstellen muss. Die Anforderungen sind unterteilt in \emph{FreeDesign-Webapp}, \emph{AI-Design-Converter-CLI} sowie \emph{SVG-Page-Renderer-CLI}.
% Der Teil \emph{FreeDesign-Webapp} bezieht auf die Webanwendung des FreeDesign-Editor und ermöglicht somit die Definition weiter grafischer Oberflächen des FreeDesign Editors. Denkbar hierfür wäre eine separate Oberflächen für native\footnote{Dies wäre durch die Nutzung von ReactNative möglich. \url{https://reactnative.dev}} Anwendung die sich an eine mobile Endgeräte richten. 
% \begin{filecontents}[overwrite]{\jobname-fa.tex}
%     \begin{longtable}{r@{\hspace{3mm}}lX}
%     \caption{Auflistung funktionaler Anforderungen}\\
%     \label{table:fa}
%     \rownumbereset
%     & \textbf{Anforderung} & \textbf{Beschreibung} \\
%     \hline
%     \hline
%     \multicolumn{3}{l}{\textbf{FreeDesign-Webapp}} \\
%     \hline
%     \rownumber & Darstellung Produktseite & Die Darstellung einer Produktseite ist abhängig vom Produkt und beinhaltet das Darstellen produktspezifischer Eigenschaften. \\
%     \rownumber & Darstellung Designs & Ein Design wird innerhalb einer ein Produktseite darzustellen. \\
%     \rownumber & Mehrsprachigkeit & Die Texte der grafische Oberfläche müssen für eine gegebene Landesprache übersetzt sein. \\
%     \end{longtable}
% \end{filecontents}
% \LTXtable{\linewidth}{\jobname-fa.tex}

% \subsubsection{Nichtfunktionale Anforderungen}
% \begin{filecontents}[overwrite]{\jobname-nfa.tex}
%     \begin{longtable}{r@{\hspace{3mm}}lX}
%     \caption{Auflistung der nichtfunktionalen Anforderungen}\\
%     \label{table:fa}
%     \rownumbereset
%     & \textbf{Anforderung} & \textbf{Beschreibung} \\
%     \hline
%     \hline
%     \multicolumn{3}{l}{\textbf{FreeDesign-Webapp}} \\
%     \hline
%     \rownumber & Darstellung Produktseite & Die Darstellung einer Produktseite ist abhängig vom Produkt und beinhaltet das Darstellen produktspezifischer Eigenschaften. \\
%     \rownumber & Darstellung Designs & Ein Design wird innerhalb einer ein Produktseite darzustellen. \\
%     \rownumber & Mehrsprachigkeit & Die Texte der grafische Oberfläche müssen für eine gegebene Landesprache übersetzt sein. \\
%     \end{longtable}
% \end{filecontents}
% \LTXtable{\linewidth}{\jobname-nfa.tex}

% Step 1. Anforderungen festlegen

