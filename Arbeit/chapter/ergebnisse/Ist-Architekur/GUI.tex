% GUI.tex
% Status Zweitkorrektur
\subsection{Grafische Oberfläche}

Zentral verantwortlich für die Darstellung der grafischen Oberfläche sind die \emph{Container}-Komponenten im Ordner \lstinline|containers|. Wie bereits in  Abschnitt \ref{sec:overview} erläutert, nutzten die Container für die Darstellung ReactJS-Komponenten aus dem Ordner \lstinline|components|.
Diese Komponenten sind somit nicht an den \emph{Redux-State} gebunden und rufen keine \emph{Redux-Actions} auf.
In der Tabelle \ref{table:compMap} wird der Quelltext der Ordner \lstinline|containers| und \lstinline|components| Bausteinen zugeordnet. Die Abhängigkeiten der grafischen Oberfläche sind sehr umfangreich und lassen sich nicht sinnvoll Abbilden. Sie folgen jedoch stets dem selben Muster, welches durch die Abbildungen \ref{fig:obersteOrdnerebene} und \ref{fig:Redux} dargestellt wird.

\begin{filecontents}[overwrite]{\jobname-con-map.tex}
\begin{longtable}{r@{\hspace{5mm}}llX}
\caption{Zuordnung des Quelltext der Ordner \lstinline|containers| und \lstinline|components| zu Bausteinen} \\
\label{table:compMap}
\rownumbereset
& \textbf{Baustein} & \textbf{Quelltexte} \\
\hline

\hline
\rownumber & Anwender-Menü 
& containers/burgerMenu/* \\
\hline

\rownumber & Anwenderregistrierung 
& containers/socialMedia/* \\
& & containers/registerDialog/* \\
\hline

\rownumber & Bestätigungsdialog 
& containers/confirmDialog/ \\ 
\hline

\rownumber & Bild-Upload 
& containers/common/DragAndDropUpload.tsx \\
& & containers/loadDesignImageDialog/* \\
\hline 

\rownumber & Bildmaskeneditor 
& containers/imageMaskEditor/* \\
\hline 

\rownumber & Design-Migration 
& containers/customerDesignMigration/*\\ 
\hline 

\rownumber & Designbearbeitung 
& containers/stage/* \\
& & components/svgRenderers/pageAssets/\\ 
& & \>EmptyImageBox.tsx \\
\hline

\rownumber & Designdarstellung 
& components/svgRenderers/\\ 
& & \>SVGTextRenderer.tsx \\
& & \>SVGImageRenderer.tsx \\
& & \>pageAssets/BackgroundImage.tsx \\
& & \>SVGItemRenderer.tsx \\
& & \>SVGDefsRenderer.tsx \\
& & \>SVGPageRenderer.tsx \\
& & \>index.ts \\
& & components/designPresenter/\\ 
& & \>DesignPresenter.tsx \\
\hline 

\rownumber & Designelementauswahl 
& containers/selectionLayer/* \\
\hline 

\rownumber & Designgalerie 
& containers/customerDesignsGallery/\\
\hline 

\rownumber & Dialogfenster 
& containers/modalContainer/ModalContainer.tsx \\
\hline 

\rownumber & Fotogalerie 
& containers/photoGallery/* \\
& & containers/customerPhotoGallery/* \\
\hline 

\rownumber & Fehlerdialog 
& containers/crashReport/* \\
\hline 

\rownumber & Fußleiste 
& containers/Footer.tsx \\
\hline 

\rownumber & Grafische-Oberfläche 
& components/* \\
& & components/controller/\\ 
& & \>rangeInputBox/RangeInputBox.tsx \\
\hline 

\rownumber & Kalenderkonfigurator 
& containers/calendarDialog/* \\
\hline 

\rownumber & Kopfleiste 
& containers/Header.tsx \\
\hline 

\rownumber & Kurzinformation 
& containers/tooltip/* \\
\hline 

\rownumber & Produktdarstellung 
& components/pagePresenter/* \\
& & components/svgRenderers/pageAssets/\\ 
& & \>PageLabels.tsx \\
& & \>GutterLines.tsx \\
& & \>BleedLines.tsx \\
& & \>SVGNonPrintAreaRenderer.tsx \\
& & \>FoldLines.tsx \\
& & components/svgRenderers/\\ 
& & \>SVGProductImageRenderer.tsx \\
\hline 

\rownumber & Produktkonfigurator 
& containers/checkoutBox/* \\
\hline 

\rownumber & Referenzwerkzeug 
& containers/reference/* \\
\hline 

\rownumber & Speicherkomponente 
& containers/changeDesignNameDialog/* \\
\hline 

\rownumber & Tastatur-Kurzbefehle 
& containers/common/Shortcuts.tsx \\
\hline 

\rownumber & Warenkorb 
& containers/shoppingCart/* \\
\hline 

\rownumber & Werkzeugmenü 
& containers/menues/toolbox/* \\
\hline 

\end{longtable}
\end{filecontents}
\LTXtable{\linewidth}{\jobname-con-map.tex}


% Die grafische Oberfläche bindet den Baustein zum Designbearbeitung und damit auch die Bausteine Design- und Produktdarstellung ein. Um diese Bausteine herum sind verschiedenste grafische Elemente integriert, die das Bearbeiten des Design unterstützen, sowie die Produktkonfiguration. Außerdem sind Elemente zur steuern  und visualisieren von Kundendaten enthalten, wie das Speicher und Öffnen von Designentwürfen oder die Anzeige des Warenkorbs.
% Die Abhängigkeiten der grafischen Oberfläche sind sehr komplex und sind in der Darstellung \emph{D3\_Grafische\_Oberfläche.html} visualisiert. 

% Die grafische Oberfläche ist auch das volatilste Element des FreeDesign-Editors und unterliegt ständiger Änderungen und Erweiterungen. Basierend auf Martin (\citeyear[119]{Martin2018}) sollte der volatile Teile eines Quelltext von stabilen Teile isoliert werden. Gerade im Quelltext für die Bereitstellung der Funktionalitäten zu Verschieben, Rotieren und Skalieren von Designobjekt besteht die Gefahr, dass Änderungen der Grafischen Darstellung dem stabile Quelltext für Objekt-Transformationen beschädigt. 
% Folgende Bausteine der grafischen Oberfläche wurden identifiziert.

% \begin{multicols}{2}    
%     \begin{enumerate}
    \item API-Kommunikation
    \item Anwender-Menü
    \item Anwenderregistrierung
    \item Bestätigungsdialog
    \item Bild-Upload
    \item Bildmaskeneditor
    \item Bildverarbeitung
    \item Cache
    \item Cookie-Verarbeitung
    \item Design-Parser
    \item Designbearbeitung
    \item Designdarstellung
    \item Designelementauswahl
    \item Designgalerie
    \item Designobjekt-Transformation
    \item Designobjekterzeugung
    \item Designstruktur
    \item Dialogfenster
    \item Farbstruktur
    \item Fehlerdialog
    \item Formularverarbeitung
    \item Fotogalerie
    \item Grafische-Oberfläche
    \item JavaScript-Erweiterung
    \item Kalendariumerzeuger
    \item Kopfleiste
    \item Kurzinformation
    \item Mathematik
    \item Maßeinheit-Konverter
    \item Produktdarstellung
    \item Produktkonfigurator
    \item Produktseiten-Bild-Konverter
    \item Produktstruktur
    \item Referenzwerkzeug
    \item SVG-Parser
    \item Schriftverarbeitung
    \item Speicherkomponente
    \item Startvideo
    \item Tastatur-Kurzbefehle
    \item Textschlüsselsammlung
    \item URL-Verarbeitung
    \item Warenkorb
    \item Werkzeugmenü
    \item XML-Parser
    \item Zwischenablageschnittstelle
\end{enumerate}
% \end{multicols}