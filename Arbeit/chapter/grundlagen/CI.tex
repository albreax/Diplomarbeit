% CI.tex
\section{Integration des FreeDesign-Projektes}

\subsection{Veröffentlichungszyklus}
Die Geschäftsführung des Unternehmens gibt einen wöchentlichen Veröffentlichungszyklus für das gesamte Softwaresystem vor, wobei jeden Montag am frühen Morgen eine neue Version veröffentlicht wird. 

Dies wird durch die Anwendung des Paradigmas der kontinuierlichen Integration ermöglicht. 
Das Paradigma wurde durch Kent Beck (\citeyear[59]{Beck2004}) geprägt und beschreibt, dass die Entwickler eines Team regelmäßig Änderungen in die Quelltextbasis integrieren. Nach jeder Integration wird werden automatisiert das ausführen der Unit-Test sowie das Kompilieren der Software durchgeführt. Durch diesen Prozess erhalten die Entwickler ein regelmäßigen Feedback über die Richtigkeit ihrer Entwicklung.

\subsection{Git-Struktur}
Der FreeDesign-Editor für die produktive Anwendung wird auf der Basis des Haupt-Branch namens \emph{master} kompiliert. 
Jeden Montag wird durch die Zusammenführung mit einem \emph{release}-Branch die Veröffentlichung der neuen Version durchgeführt. 
Der \emph{release}-Branch wiederum wird in der Woche zuvor donnerstag Vormittag auf der Basis eines \emph{develop}-Branch erstellt. Neue Funktionalitäten, Fehlerkorrekturen und sonstige Änderungen am Quelltext werden in eigenen Git-Branchen, sogenten \emph{Feature}-Branches entwickelt. Wurde eine Änderung fertiggestellt, wird mit einem \emph{Pull-Request} die Zusammenführung der Änderung in den \emph{develop}-Branch beantragt.
Wurde die Zusammenführung genehmigt und durchgeführt, ist die Änderung somit automatisch Bestandteilt der nächsten Veröffentlichung. 


\subsection{Bereitstellungsablauf}
Vor der Zusammenführung wird jedoch die Änderung intensiv geprüft. Hierfür wird automatisch auf der Basis des \emph{Pull-Requests}
eine Pipeline in Bitbucket ausgelöst. 
Basierend auf der Beschreibung von Silvio Gohl (Operations-Team Unitedprint.com SE) kann der Ablauf der Pipeline wie folgt beschrieben werden.
Zunächst erstellt die Pipeline auf der Basis des Feature-Branch einen temporären Branch, in welchem der Feature-Branch und der \emph{develop}-Branch zusammengeführt werden. 
Die nächsten Schritte der Pipeline sind die Ausführung der Unit-Tests und das testweise Kompilieren des FreeDesign-Editors. Treten bei diesen Schritten Fehler auf, müssen sie durch den Entwickler korrigiert und der Pull-Request neu erstellt werden. 

Verlief das Erstellen des Pull-Request ohne Probleme, werden die anderen Mitglieder des Teams informiert und gebeten eine Begutachtung der Änderungen durchzuführen. Diese haben nun die Möglichkeit die Änderungen in Bitbucket zu begutachten und zu kommentieren. Wurden die Änderungen von mindestens zwei Entwicklern bestätigt, kann der Feature-Branch mit dem develop-Branch zusammengeführt werden. 

Sobald eine Änderung im develop-Branch enthalten ist, steht sie auf einem Testserver zur Verfügung. Die Änderung wird nun von der hausinternen Abteilung für die Qualitätssicherung, sowie von der Fachabteilung, die die Änderung in Auftrag geben hat, überprüft.