% Erstprüfung
\section{Refactoring}

Basierend auf der Definition von Martin Fowler (\citeyear[S. 79-80]{Fowler2020}), wird in der Softwareentwicklung das Umstrukturieren von Quelltext einer Software als Refactoring bezeichnet. Die Änderung der Struktur verursacht jedoch keine Änderung des äußeren Verhaltens der Software, sondern nur eine Verbesserung des Designs des Quelltextes.
Beim Durchführen des Refactoring wird in kleinen Schritten vorgegangen, sodass der Zustand vermieden wird, in dem die Software nicht funktioniert.  

Das Refactoring von Quelltext gehört zu den regelmäßigen Tätigkeiten des Entwicklerteams des FreeDesign-Projektes. Die Gründe hierfür sind unteranderem das Vorbereiten der Umsetzung neuer Funktionalitäten, das Vermeiden der mehrfachen Implementation von gleicher Logik oder das Optimieren des Quelltextes auf der Basis von aktuellem Wissens.

Um Fehler durch die Änderungen zu vermeiden, ist es wichtig, dass betroffener Quelltexte durch Unit-Tests überprüft wird und somit die Erhaltung der äußeren Funktionalität sichergestellt werden kann \autocite[vgl.][211]{ThomasAndHunt2020}. 
