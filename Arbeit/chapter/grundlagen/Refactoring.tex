\section{Refactoring}

Basierend auf Fowler (\citeyear[S. 79-80]{Fowler2020}) wird in der Softwareentwicklung als Refactoring, das Umstrukturieren von Quelltexte einer Software bezeichnet. Die Änderung der Struktur verursacht jedoch keine Änderung des Verhaltens der Software, sondern nur eine Verbesserung des Designs des Quelltextes.
Der Ansatz des Refactoring verfolgt dabei den Ansatz, den Quelltexte in kleinen Schritten zu ändern, sodass Zustand vermieden wird, in dem Software nicht funktioniert.  

Um Fehler durch das Refactoring zu vermeiden, ist es wichtig, dass betroffener Quelltexte durch Unit-Tests abgedeckt ist und somit die Erhaltung der Funktionalität überprüft werden kann. 
