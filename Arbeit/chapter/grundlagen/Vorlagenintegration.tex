% Status: Erstprüfung
\section{Integrationsprozess der Designvorlagen}
Auf dem Webportal \emph{design.easyprint.com} wird für jedes Produkt nur ein Produktformat zur Gestaltung einer Designvorlage angeboten. Durch einen Integrationsprozess wird die eigenreichte Designvorlage auf die anderen Formate des Produktes sowie auf weitere Produkte adaptiert.

Die Adaptierung führt eine hauseigene Grafikabteilung unter verwendung der Software \emph{Adobe Illustrator}, welche eine Software zur Erstellung von Vektor-Grafiken ist \autocite[vgl.][]{Adobe:Illustrator}. 
Ein Teil der Adaptierungsarbeit ist das Definieren von Text- und Farbschlüsseln, die später im Integrationsprozess durch Übersetzungstexte und Farben der Farbschemen ersetzt werden. 

Die Designvorlagen werden im Anschluss als SVG-Dateien exportiert und durch einen Importprozess in das Easyprint-Shop-System integriert. 
In einem ersten Schritt wird zunächst die SVG-Datei in eine Datenstruktur übersetzt, welche der FreeDesign-Editor einlesen kann. Auf Basis der so entstandenen  
Datenstruktur werden die Sprach/Farb-Derivate erzeugt und auf den Webservern, für die Verwendung im FreeDesign-Editor, gespeichert.  

In einem letzten Schritt wird das 3D-Vorschaubild erzeugt, welches zur Präsentation auf der Übersichtsseite eingesetzt wird. Für das Erzeugen eines Vorschaubildes werden die SVG-Grafiken der Designs der einzelnen Produktseiten verwendet. 
Diese werden in einem Schritt vorher, direkt nach dem Speichern der Sprach/Farb-Derivate erzeugt. Es wird somit für jedes Sprach/Farb-Derivat ein 3D-Vorschaubild erzeugt.
Die Abbildung \ref{fig:Vorlagenimport} stellt den gesamten Integrationsprozess durch ein Aktivitätsdiagramm dar.

\begin{figure}[H]
    \centering
    \includegraphics[width=.98\textwidth]{diagrams/FreeDesign-Vorlagenerstellung.pdf}
\caption{Integrationsprozess der Designvorlagen als Aktivitätsdiagramm}
\label{fig:Vorlagenimport}
\end{figure}

Das FreeDesign-Projekt unterstützt den Integrationsprozess durch die Bereitstellung eines Kommandozeilenprogramms für die automatische Konvertierung der SVG-Dateien in die Datenstruktur für den FreeDesign-Editor. Weiterhin wird ein Kommandozeilenprogramm für die automatische Erzeugung der SVG-Dateien der einzelnen Produktseiten bereitgestellt. Beide Schritte sind in Abbildung \ref{fig:Vorlagenimport} gestrichelt eingerahmt. 