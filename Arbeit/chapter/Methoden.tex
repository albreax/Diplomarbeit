\chapter{Methoden}

\section{Architektur-Workshop}
Jedes Mitglied des Entwicklerteams des FreeDesign-Projektes hat tiefgreifende Erfahrungen mit der Architektur des Quelltextes gesammelt. 
Die Kenntnisse über den Quelltext sind nicht gleich über das Team verteilt und unterscheiden sich, je nachdem an welchem Teilen der Software die einzelnen Mitglieder in der Vergangenheit gearbeitet haben.
Durch Erfahrungen und Kompetenzen kann jedes Mitglied des Teams wertvoll Hinweise zu Problemen und Schwächen der aktuellen Architektur geben, sowie Ideen und Wünsche für eine Überarbeitung der Architektur beitragen.
Um diese Hinweise zu sammeln und zu diskutieren, wurde vor der eigentlichen Analyse ein Architektur-Workshop mit dem Entwicklerteam des FreeDesign-Editors durchgeführt. Neben den Schwächen der Ist-Architektur wurden auch Stärken hervorgehoben, die es zu bewahren gilt.

Auch der Beitrag eines neuen Teammitgliedes erwies sich als sehr hilfreich, da dieses eine frische Sicht auf das Projekt hatte. Es konnten dadurch Erfahrungen gesammelt werden, wie gut der Einstieg in die Projektstruktur gelingt. 

Damit das Entwicklerteam sich optimal auf den Workshop vorbereiten konnten, wurde er zwei Wochen vorher angekündigt. Eine Woche vor dem Workshop erhielten die Teammitglieder eine Agenda mit Fragen zur Architektur:
\begin{itemize}
	\item Was ist positiv in der aktuellen Architektur hervorzuheben?
	\item Welche Schwächen oder Probleme werden in der aktuellen Architektur gesehen?
	\item Welche Ideen und Vorschläge sollen in eine Überarbeitung der Architektur beachtet werden?
\end{itemize}

% ===================================================== % 

\section{Analyse der statischen Ist-Architektur}
\subsection{Reverse Engineering}
Da die Ist-Architektur des FreeDesign-Editors kaum dokumentiert war, musste die aktuelle Architektur zunächst rekonstruiert werden. 
Dies wird als \emph{Reverse Engineering} bezeichnet und hat, laut der Definition von Chikofsky und Cross (\citeyear[S. 13-17]{Chikofsky1990}), zwei Ziele: 
\begin{itemize}
    \item Die Darstellung des Softwaresystem in einer abstrakten Form. 
    \item Die Identifikation der Komponenten eines Softwaresystems und ihre Beziehungen untereinander. 
\end{itemize}

% Für das Reverse Engineering der statischen Ist-Architektur stand im Fokus, eine Grundlagen für den Entwurf einer statischen Soll-Architektur zu erstellen.
% Daher konzentriete sich die Reverse-Engineering auf das identifizieren von Baustein aus denen der FreeDesign-Editor besteht und auf denen die Soll-Architektur basieren kann. 
% Den Bausteinen wurden weiterhin Bestandteile des Quelltextes der aktuellen Implementation zugeordnet. 

\subsection{Darstellung des Softwaresystems}
Das Reverse Engineering kann durch den Einsatz von Analyse-Werkzeugen unterstützt werden, wobei üblicherweise einzelnes Werkzeug nicht alle Analyse-Aufgaben übernehmen können \autocite[vgl.][381]{Bass2013}.  

Für die Visualisierung der Ist-Architektur wurde das Werkzeug \emph{dependency-cruiser} eingesetzt, welches das Quellentextverzeichnis eines TypeScript-Projektes untersucht und die Struktur des Quelltext als Abhängigkeitsgraph visualisiert. Weiterhin können Regeln für die Abhängigkeiten der Komponenten angelegt werden, die durch das Werkzeug validiert, Verstöße gekennzeichnet und als Report ausgeben werden \autocite[vgl.][]{Verweij:Dependency}. 

Durch das nutzen verschieden Aufrufparameter konnten verschieden Darstellung, mit unterschiedlichem Fokus, erzeugt werden.
Zunächst wurde die Beziehungen des Quelltextkomponenten auf der obersten Ordnerebene visualisiert und analysiert. 
Weitere Analysen bezogen sich auf die Hauptaufgaben der Software:
\begin{itemize}
    \item die Darstellung von Produkten
    \item die Darstellung von Designs
    \item das Editieren von Designs
    \item die grafische Oberfläche des FreeDesign-Editors
    \item das Konvertieren von Adobe-Illustrator-Dateien
    \item das Konvertieren von FreeDesign-Designs in SVG-Dateien
\end{itemize}

% Das Werkzeug dependency-cruiser unter einer MIT-Linzenz zu Verfügung gestellt, welche eine kostenfreien Nutzung garantiert \autocite[vgl.][]{Verweij:License}. 


\subsection{Identifikation von Bausteinen}
Starke und Hruschka bezeichnen die Komponenten eines Softwaresystems als Bausteine, die miteinander in Beziehungen stehen \autocite[vgl.][24]{Starke2011}. Diese Bezeichnung wird auch in der vorliegenden Diplomarbeit verwendet, um einer Verwechselung zwischen Softwarebausteinen und React-Komponenten vorzubeugen.

Mit Hilfe der Abhängigkeitsgraphen wurden die Bausteine, aus denen der FreeDesign-Editor besteht, identifiziert. 

Um Fragmente des Quelltextes den Bausteinen zuzuordnen, wurde ein kleines Werkzeug entwickelt, welches das Quellenverzeichnis rekursiv durchsucht. Zum einen sucht es nach Dateien mit der Bezeichnung \glqq\lstinline|_component.md|\grqq{} und zum anderen, innerhalb der Quelltextdateien, nach der Textphrase \glqq\lstinline|// @component{BAUSTEINNAME}|\grqq{}. 
Mit Hilfe der \lstinline|_component.md|-Dateien, deren einziger Inhalt der Name eines Bausteins ist, kann der Inhalt eines gesamten Verzeichnisses einem Baustein zugeordnet werden. Einzelne Quelltextdateien können mit der Textphrase \glqq\lstinline|// @component{BAUSTEINNAME}|\grqq{} Bausteinen zugeordnet werden. Die Textphrase kann auch mehrfach innerhalb einer Datei enthalten sein, wodurch es möglich ist, ein Datei mehreren Bausteinen zuzuordnen. 

Das Ergebnis der Analyse ist eine tabellarische Zuordnung von Bausteinen zu Quelltext-Fragmenten in Form einer HTML-Datei. 

Das Werkzeug ist im Anhang unter \emph{Komponentenzuordnung} enthalten.
% Mit dem Reverse Engineering einer Architektur sollten klar definierte Ziele erreicht werden \autocite[vgl.][200]{Reussner2009}. 
% Für diese Arbeit war das primäre Ziel des Reverse Engineerings die Vorbereitung des Entwurfs der statischen Soll-Architektur. 


% Ein Softwaresystem setzt sich aus unterschiedlichen Bestandteilen zusammen wie Klassen, Funktionen, Module und Schnittstellen, sowie Beziehungen zwischen ihnen. Basierend auf Starke (\citeyear[S. 24]{Starke2011}) können diese Bestandteile auch generisch als \emph{Bausteine} bezeichnet werden.

% Somit war ein Ziel des Reverse Engineering das identifizieren von Bausteinen aus dem der FreeDesign-Editor besteht. 
%Weitere Ziele waren den Bausteinen zugehörige Quelltext-Bestelltandteile der aktuelle Implementation zuzuordnen, sowie die Abhängigkeiten der Quelltext-Bestelltandteile darzustelegen. 

%Um die Ziele zu erreichen wurden zun nächst die Beziehungen des Quelltext auf der obersten Ordnerebene analysiert. Hierfür wurde das Werkzeug 



% Die Analyse konzentriete sich dabei auf die Aspekte, die im Architektur-Workshop vom Entwicklerteam herausgearbeitet wurden.
% Für die Rekonstruktion standen folgende Informationsquellen zur Verfügung:
% \begin{itemize}
% 	\item Das Fachwissen des Entwickler-Teams
% 	\item Der Quelltext des Projektes
% 	\item Projektdokumente (Confluence), welchen verschiedener Funktionalitäten des Editors beschreiben. 
% 	\item Das Log der Versionsverwaltung (Git)
% 	\item Das Ticketsystem (Jira), über welches die Entwicklungsaufgaben verwaltet werden. 
% \end{itemize} 

% Das Reverse Engineering kann durch den Einsatz von Analyse-Werkzeugen unterstützt werden, wobei üblicherweise nicht einzelnes Werkzeug alle Analyse-Aufgaben übernimmt \autocite[vgl.][381]{Bass2013}. 

% Nach dem Erörtern der Aspekte, wurden geeignete Werkzeuge ermittelt, welche die Arbeit des Reverse Engineering unterstützen können.

\section{Entwurf der Soll-Architektur}

% Komponenten aus Ist-Architektur in volatile und stabile Komponenten unterscheiden => Clean Architecture	 
