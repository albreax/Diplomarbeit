% Diskussion.tex

%TODO: Diskussion
%TODO: Auf fehlendes SOLID eingehen
%TODO: Keine kohesion in common Ordnern
%TODO: Ansätze von Smart-UI
%TODO: Ist-Archtitektur eher Package by Layer-Architektur auf oberste eben => Package by component darunter
% Architekturworkshop => unterschiedlich gute Motivation der Teammitglieder
% Analyse Ist-Architektur  => es wurden Artefakte in Quelltext eingeführt

%\paragraph{Kritik }
%Eine gute Architektur zeichnet sich, basierend auf Robert C. Martin (\citeyear[S. 196 - 197]{Martin2018}), dadurch aus, dass sie keine Vorgaben über die verwendete Technologien macht und offen gebgenüber Technologieänderungen ist. 
%Diese wird von der aktuellen Architektur auf der obersten Ordnerebene nicht erfüllt. Würde z.B. ein Technologwechsel zur Verwaltung des Anwendungszustands durchgeführt werden, würde das einen Änderung der Architektur bedeuten.

% Mit  1260 Quelltextzeilen, 51 Funktionen und 46 Importen, handelt es sich um die komplexeste Komponente des FreeDesign-Editor und kann als eine \emph{Man-in-the-middle-Klasse} bezeichnet werden. Hierbei handelt es sich um ein Antimuster bei dem eine Klasse zu viele zuständigkeiten hat \autocite[vgl.][619]{Geirhos2015}.

%
%Der aktuellen Architektur wurde zum Zeitpunkt ihrer Entwicklung kein Architekturmuster zugrunde gelegt, was viel Raum zu Interpretation der Architektur lässt. 

% TODO: Diskussion Aufteilung der Redux-Files
% TODO: Es war schwierig Diagramme zu erstellen, da keine klasse / interfaces

% TODO: auf Quelltextleichen eingehen

% \paragraph{Testbarkeit} 
% TODO: Bei Änderung des Quelltextbasis müssen sowohl der FreeDesign-Editor als die Kommandozeilenprogramme getestet werden, auch wenn die Änderung nur ein Programm betrifft.


%TODO: \paragraph{Gefahr von Quelltextleichen}
%% Unter Quelltextleichen ist Quelltext zu verstehen, der nicht ausgeführt wir \autocite[vgl.][292]{Martin2009}. Sollte das Importieren von Designvorlagen nicht mehr notwendig sein oder es wird ein alternativer Prozess gefunden, existiert weiterhin Quelltext im FreeDesign-Projekt, der das Projekt aufbläht und aufwändig entfernt werden sollte.
