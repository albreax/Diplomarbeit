% Status zweit Korrektur
\section{Architektur-Workshop}
Das Entwicklerteam des FreeDesign-Editors bestand (Februar 2021) aus acht Teammitgliedern. Jedes Mitglied hat tiefgreifende Erfahrungen mit der Quelltextarchitektur des Projektes gesammelt. 
Die Kenntnisse über den Quelltext sind jedoch nicht gleich über das Team verteilt und unterscheiden sich, je nachdem an welchen Teilen der Software die einzelnen Mitglieder in der Vergangenheit gearbeitet haben.
Durch Erfahrungen und Kompetenzen kann jedes Mitglied des Teams wertvolle Hinweise zu Problemen und Schwächen der aktuellen Architektur geben, sowie Ideen und Wünsche für eine Überarbeitung der Architektur beitragen.

Diese Hinweise sollten gesammelt und diskutiert werden, um sie bei einem Entwurf einer Soll-Architektur zu beachten. Um den zeitlichen Aufwand für die Mitlieder so klein wie möglich zu halten, wurde ein zweistündiger Workshop entwickelt und durchgeführt. Dieser hatte zum Ziel, anhand von Fragestellungen eine offene Diskussion über die Ist-Architektur sowie über die Ideen und Wünsche für eine Soll-Architektur zu ermöglichen. 
Hierzu wurden, zur Strukturierung der Diskussion, folgende drei Fragen formuliert: 
\begin{itemize}
	\item Was ist positiv in der aktuellen Architektur hervorzuheben?
	\item Welche Schwächen oder Probleme werden in der aktuellen Architektur gesehen?
	\item Welche Ideen und Vorschläge sollen in eine Überarbeitung der Architektur beachtet werden?
\end{itemize}
Damit das Entwicklerteam sich optimal auf den Workshop vorbereiten konnte, wurde dieser zwei Wochen vorher angekündigt. Die Fragen wurden den Mitgliedern des Teams eine Woche im Voraus zugesandt, mit der bitte sich Gedanken über die Fragestellungen zu machen.  

Um der Gefahr vorzubeugen, dass die Diskussion ihren Fokus verliert, wurde sie klar auf das Thema Architektur eingeschränkt.
Hierfür wurde das Aufgreifen von Themen wie beispielsweise die konkrete Implementation von spezifischen Funktionalitäten, den Einsatz bestimmter Technologien oder die Umsetzung von Projekten ausgeschlossen. Während der Diskussion wurde diese Regel durchgesetzt, indem eine solche Diskussion unterbrochen wurde und auf das eigentliche Thema verwiesen wurde. Der diskutierte Punkt wurde jedoch für spätere Diskussionen festgehalten.
