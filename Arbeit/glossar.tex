% \chapter*{Glossar}
% \addcontentsline{toc}{chapter}{Glossar}
% \begin{description}
%     \item[Freedesign-Editor] Eine Software zur Gestaltung von Druckprodukten, die von der Onlinedrucker Unitedprint.com SE entwickelt wird und in den Onlineshops des Unternehmens bereitgestellt wird.
%
%     \item[Unit-Testing] Unit-Testing ein Vorgehen zum Testens des Quelltextes einer Software.
% \end{description}

\longnewglossaryentry{Freedesign-Editor}{name=Freedesign-Editor}{Eine Software zur Gestaltung von Druckprodukten, die von der Onlinedrucker Unitedprint.com SE entwickelt wird und in den Onlineshops des Unternehmens bereitgestellt wird.}

\longnewglossaryentry{Unit-Testing}{name=Unit-Testing}{Unit-Testing ein Vorgehen zum Testens des Quelltextes einer Software.}

\longnewglossaryentry{Unit-Test}{name=Unit-Test}{Der Unit-Test ist ein Test zur Prüfung von Quelltext.}
